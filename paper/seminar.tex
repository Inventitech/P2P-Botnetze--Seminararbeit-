\documentclass{llncs}

%% Verwende A4-Format statt Letter
\usepackage{a4}
%% Deutsche Silbentrennung und Sprache (neue Rechtschreibung)
\usepackage[ngerman]{babel}
%% Verwende Schriftart mit "echten" Umlauten statt Akzenten
\usepackage[T1]{fontenc}
%% Verwende Umlaute direkt
\usepackage[utf8x]{inputenc}
%% Hyperlinks für interne Referenzen
\usepackage{hyperref}
%% Grafiken einbinden
\usepackage{graphicx}
%% Paket für Unterabbildungen pro Abbildung
%\usepackage{subfig}

% Titel der Arbeit
\title{Peer-To-Peer in Botnets}

% Angaben zum Author
\author{Moritz Marc Beller,\\Ben Nachname}
\institute{%
   Fakultät für Informatik, \\
   Technische Universität München \\
%    Munich, Germany\\
   \email{\{beller,bennachname\}@in.tum.de}
}

\pagestyle{plain}

%------------------------------------------------------------------------------
\begin{document}

\maketitle

%------------------------------------------------------------------------------
\begin{abstract}
Diese Arbeit behandelt ein interessantes Thema.
\end{abstract}

%------------------------------------------------------------------------------
\section{Einleitung}



\section{Definitions}
A computer able of executing remotely-triggered commands is called a {\it bot} or {\it zombie.} A {\it botnet} is a group of bots forming a common network structure.\cite{schoof2007detecting} In most recent papers on the subject (\cite{wang2009systematic}, \cite{abu2006multifaceted}), the term botnet is defined as purely negative, i.e. a network performing destructive aims such as DDoS attacks, sending spam or hosting a phishing website\cite{steggink2007detection}. We'd like to propose a bias-free definition of botnet as per our understanding technology is generally ethics-free. Additionally, there are many examples where botnets are used in a non-destructive way (e.g. \cite{seti}), or even to destroy existing ``evil-minded'' botnets.

A {\it botmaster} is referred to as the controller of the botnet. This doesn't necessarily have to be the founder of the botnet.

The expression {\it bot candidates} specifies the set of computers
which are target to becoming a bot themselves.

{\it Peer-to-Peer}, being a technology buzz word of the internet in
the late 1990s with file sharing services like Napster\cite{napster},
has attracted less attention in recent years. {\it P2P } defines an
unstructured information network amongst equals --- so-called
peers. Two or more peers can spontaneously exchange information
without a central instance. According to \cite{schoder2005core} ``P2P
networks promise improved scalability, lower cost of ownership,
self-organized and decentralized coordination of previously underused
or limited resources, greater fault toler- ance, and better support
for building ad hoc networks.''  These properties coupled with the
fact that files circumfloating in P2P networks are prone to malware,
trojans and viruses make P2P networks a most-attractive base for
building botnets.  Well-known P2P networks include the
Napster\cite{napster}, Gnutella, Overnet and Torrent network.

The so-called {\it C\&C}, command and controll structure, specifies
the way and protocols in which the botmaster and the bots communicate
to each other. It is the central property of any botnet.

{\it IRC} --- internet relay chat --- is a ``teleconferencing
system''\cite{irc}, typically used for text chatting in channels
joined by a large number of participants. While its protocol is
relatively easy to implement, it provides a lot of features. It has
thus become the de-facto standard for conventional botnets.

The process of {\it bootstrapping} generally describes starting a more
complex system ontop of a simple system. In regard to botnets, the
term usually means loading of the bot code (often injected into the
original filesharing program) and establishing a connection to other
bots.\cite{wang2009systematic}



%------------------------------------------------------------------------------
\bibliographystyle{alpha}
\bibliography{literature}



\end{document}
