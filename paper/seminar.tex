\documentclass{llncs}

%% Verwende A4-Format statt Letter
\usepackage{a4}
%% Deutsche Silbentrennung und Sprache (neue Rechtschreibung)
\usepackage[ngerman]{babel}
%% Verwende Schriftart mit "echten" Umlauten statt Akzenten
\usepackage[T1]{fontenc}
%% Verwende Umlaute direkt
\usepackage[utf8x]{inputenc}
%% Hyperlinks für interne Referenzen
\usepackage{hyperref}
%% Grafiken einbinden
\usepackage{graphicx}
%% Paket für Unterabbildungen pro Abbildung
%\usepackage{subfig}

% Titel der Arbeit
\title{Peer-To-Peer in Botnets}

% Angaben zum Author
\author{Moritz Marc Beller,\\Ben Nachname}
\institute{%
   Fakultät für Informatik, \\
   Technische Universität München \\
%    Munich, Germany\\
   \email{\{beller,bennachname\}@in.tum.de}
}

\pagestyle{plain}

%------------------------------------------------------------------------------
\begin{document}

\maketitle

%------------------------------------------------------------------------------
\begin{abstract}
Diese Arbeit behandelt ein interessantes Thema.
\end{abstract}

%------------------------------------------------------------------------------
\section{Einleitung}



\section{Definitions}
A computer able of executing remotely-triggered commands is called a
{\it bot} or {\it zombie.} A {\it botnet} is a group of bots forming a
common network structure.\cite{schoof2007detecting} In most recent
papers on the subject (\cite{wang2009systematic},
\cite{abu2006multifaceted}), the term botnet is defined as purely
negative, i.e. a network performing destructive aims such as DDoS
attacks, sending spam or hosting a phishing
website\cite{steggink2007detection}. A common aim is to provide the
aggregated CPU resources of the botnet, or stealing user's
credentials. \cite{borgaonkar2010analysis} We'd like to propose a
bias-free definition of botnet as per our understanding technology is
generally ethics-free. Additionally, there are many examples where
botnets are used in a non-destructive way (e.g. \cite{seti}), or even
to destroy existing ``evil-minded'' botnets.

A {\it botmaster} is referred to as the controller of the botnet. This
doesn't necessarily have to be the founder of the botnet (cf. \ref{ClassificP2P}).

The expression {\it bot candidates} specifies the set of computers
which are target to becoming a bot themselves.

{\it Peer-to-Peer}, being a technology buzz word of the internet in
the late 1990s with file sharing services like Napster\cite{napster},
has attracted less attention in recent years. {\it P2P } defines an
unstructured information network amongst equals --- so-called
peers. Two or more peers can spontaneously exchange information
without a central instance. According to \cite{schoder2005core} ``P2P
networks promise improved scalability, lower cost of ownership,
self-organized and decentralized coordination of previously underused
or limited resources, greater fault toler- ance, and better support
for building ad hoc networks.''  These properties coupled with the
fact that files circumfloating in P2P networks are prone to malware,
trojans and viruses make P2P networks a most-attractive base for
building botnets.  Well-known P2P networks include the
Napster\cite{napster}, Gnutella, Overnet and Torrent network.

The so-called {\it C\&C}, command and controll structure, specifies
the way and protocols in which the botmaster and the bots communicate
with each other. It is the central property of any botnet. Common
protocols for C\&C include IRC, HTTP, FTP and P2P.\cite{borgaonkar2010analysis}

{\it IRC} --- internet relay chat --- is a ``teleconferencing
system''\cite{irc}, typically used for text chatting in channels
joined by a large number of participants. While its protocol is
relatively easy to implement, it provides a lot of features. It has
thus become the de-facto standard for C\&C in conventional botnets.

The process of {\it bootstrapping} generally describes starting a more
complex system ontop of a simple system. In regard to botnets, the
term usually means loading of the bot code (often injected into the
original filesharing program) and establishing a connection to other
bots.\cite{wang2009systematic}

\section{A brief history of botnets}
It is not surprising that the first bot --- Eggdrop --- was a
non-malicious IRC bot. Its origins go back to the year 1993. However,
in April 1998 a deriviant called GT-Bot formed the first malicious
botnet, using IRC's C\&C structures. Four years later, in 2002,
Slapper was the first worm to make use of P2P for
C\&C.\cite{li2009botnet}

\section{The genesis of a P2P botnet}


\subsection{Classification P2P networks}
\label{ClassificP2P}
There are three types of P2P networks: ``parasite'', ``leeching'' and
``bot-only''.\cite{wang2009systematic} 

Parasite and leeching bots infiltrate existing P2P networks, while
``bot-only'' networks are designed as new networks. 

Parasite botnets recruit new bots only from the set of existing P2P
participants; they try to infect system inside the P2P network and
make them become bots. Due to the often illegal content distributed in
file sharing networks, they are a perfect culture medium of viruses,
malware and worms. It is thus convenient for an attacker to spread a
highly-demanded file (e.g. porn) containing the injection code
sequences of his bot. This code is then injected into the file sharing
client. Vulnerable hosts in the network are infected this way. On the
downside, this means that the spread of the bot is limited to the size
of the P2P network.

In contrast, leeching bots not only try to infiltrate systems which
are already part of the P2P network, but also systems outside of the
P2P network. Natuarally, they are bigger in size as they have to
deliver the P2P client, too. This might be more difficult to achieve
as it means that systems must unwillingly take part in the
network. Often, firewalls and port-forwarding are not properly
configured on these systems, reducing the performance of the
botnet. Leeching bots can spread through any possible measure: File
sharing, downloads on websites, email attachments and instant
messanging.

There are good reasons for either strategy: Using an existing P2P
network as a base like parasite and leeching bots do unburdens the
botmaster from setting up and building a botnet infrastructure. It
profits from the established P2P network, making use of filtering,
error-correction and encryption as far as the chosen network has
support for it. On the other hand, features are limited to the
existing P2P protocol. A specifically-built P2P bot-only network is
natuarally more tailored towards its purpose. Due to the bot-exclusive
memberships, it might be easier to shutdown as all participants can be
considered bots and there is no risk of accidentally shutting down an
innocent member.

\subsection{Lifetime of P2P botnets}
Wang et. all\cite{wang2009systematic} differentiate three stages of P2P botnets:
\begin{itemize}
\item recruiting bot members
\item forming the botnet
\item standing by for instruction
This is the actual ``operational'' phase of the botnet. Bots are awaiting instructions from their master. Instructions can either be actual commands or performing updates. In this phase, the chosen C\&C structure is essential.
\end{itemize}
It should be noted that these phases are not strictly exclusive,
e.g. during the third phase building of the botnet may well
continue. In fact, this is a typical property of any P2P network. It
is only until a critical mass of bots has proceeded past phase one and
two, that the botnet can be called operational.

\section{C\&C in P2P botnets}
Central server, hybrid, completely decentralized

\section{Comparison: Conventional bots vs. P2P bots}

\section{Counter measure against evil P2P botnets}

%------------------------------------------------------------------------------
\bibliographystyle{alpha}
\bibliography{literature}



\end{document}
